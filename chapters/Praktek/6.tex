\section{Bakti Qilan Mufid | 1174083}
\subsection{Buatlah library fungsi (file terpisah/library dengan nama NPM\textunderscore bar.py) untuk plot dengan jumlah subplot adalah NPM mod 3 + 2}
\hfill \break
Pertama-tama kita harus mengetahui hasil dari NPM mod 3 + 2 terlebih dahulu, lalu sesudah itu kita membuat subplotnya, seperti pada kode berikut:
\lstinputlisting[firstline=9, lastline=79]{src/6/1174083/Praktek/1174083_bar.py}

\subsection{Buatlah library fungsi (file terpisah/library dengan nama NPM\textunderscore scatter.py) untuk plot dengan jumlah subplot adalah NPM mod 3 + 2}
\hfill \break
Pertama-tama kita harus mengetahui hasil dari NPM mod 3 + 2 terlebih dahulu, lalu sesudah itu kita membuat subplotnya, seperti pada kode berikut:
\lstinputlisting[firstline=9, lastline=79]{src/6/1174083/Praktek/1174083_scatter.py}

\subsection{Buatlah library fungsi (file terpisah/library dengan nama NPM\textunderscore pie.py) untuk plot dengan jumlah subplot adalah NPM mod 3 + 2}
\hfill \break
Pertama-tama kita harus mengetahui hasil dari NPM mod 3 + 2 terlebih dahulu, lalu sesudah itu kita membuat subplotnya, seperti pada kode berikut:
\lstinputlisting[firstline=9, lastline=54]{src/6/1174083/Praktek/1174083_pie.py}

\subsection{Buatlah library fungsi (file terpisah/library dengan nama NPM\textunderscore plot.py) untuk plot dengan jumlah subplot adalah NPM mod 3 + 2}
\hfill \break
Pertama-tama kita harus mengetahui hasil dari NPM mod 3 + 2 terlebih dahulu, lalu sesudah itu kita membuat subplotnya, seperti pada kode berikut:
\lstinputlisting[firstline=9, lastline=79]{src/6/1174083/Praktek/1174083_plot.py}

\subsection{Keterampilan Penanganan Error}
Fungsi Penanganan error sebagai berikut:
\lstinputlisting[firstline=9, lastline=26]{src/6/1174083/Praktek/error.py}

%%%%%%%%%%%%%%%%%%%%%%%%%%%%%%%%%%%%%%%%%%%%%%%%%%%%%%%%%%%%%%%%%%%%%%%%%%%%%%%%%%%%%%%%%%%%%%%%%%%%%%%%%%%%%%%%
\section{Mochamad Arifqi Ramadhan | 1174074}
\subsection{Keterampilan Pemograman }
\subsubsection{Buatlah library fungsi (file terpisah/library dengan nama NPM/textunderscore bar.py) untuk plot dengan jumlah subplot adalah NPM mod 3 + 2}

Ini adalah fungsi untuk membuat plot bar sesuai dengan hasil modulus:
\lstinputlisting[firstline=7, lastline=44]{src/6/1174074/Praktek/1174074_bar.py}

Berikut ini cara pemanggilannya:
\lstinputlisting[firstline=7, lastline=21]{src/6/1174074/Praktek/main.py}

\subsubsection{Buatlah library fungsi (file terpisah/library dengan nama NPM/textunderscore scatter.py) untuk plot dengan jumlah subplot adalah NPM mod 3 + 2}

Ini adalah fungsi untuk membuat plot scatter sesuai dengan hasil modulus:
\lstinputlisting[firstline=7, lastline=44]{src/6/1174074/Praktek/a1174074_scatter.py}

Berikut ini cara pemanggilannya:
\lstinputlisting[firstline=23, lastline=37]{src/6/1174074/Praktek/main.py}

\subsubsection{Buatlah library fungsi (file terpisah/library dengan nama NPM/textunderscore pie.py) untuk plot dengan jumlah subplot adalah NPM mod 3 + 2}

Ini adalah fungsi untuk membuat plot pie sesuai dengan hasil modulus:
\lstinputlisting[firstline=8, lastline=52]{src/6/1174074/Praktek/a1174074_pie.py}

Berikut ini cara pemanggilannya:
\lstinputlisting[firstline=39, lastline=53]{src/6/1174074/Praktek/main.py}

\subsubsection{Buatlah library fungsi (file terpisah/library dengan nama NPM/textunderscore pie.py) untuk plot dengan jumlah subplot adalah NPM mod 3 + 2}

Ini adalah fungsi untuk membuat subplot sesuai dengan hasil modulus:
\lstinputlisting[firstline=7, lastline=44]{src/6/1174074/Praktek/a1174074_plot.py}

Berikut ini cara pemanggilannya:
\lstinputlisting[firstline=55, lastline=69]{src/6/1174074/Praktek/main.py}


\subsubsection{Keterampilan Penanganan Error}
\lstinputlisting[firstline=7, lastline=19]{src/6/1174074/Praktek/1174074_error.py}
%%%%%%%%%%%%%%%%%%%%%%%%%%%%%%%%%%%%%%%%%%%%%%%%%%%%%%%%%%%%%%%%%%%%%%%%%%%%%%%%%%%%%%%%%%%%%%%%%%%%%%%%%%%%%%%%%%%%%%%%%%%%%%%%%%