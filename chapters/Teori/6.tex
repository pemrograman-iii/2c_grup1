\section{Dini Permata Putri}
\begin{enumerate}


\item 1. Apa itu fungsi library Matplotlib
Matplotlib adalah sebuah library pada python yang digunakan untuk membuat diagram. Library ini biasanya menghasilkan ploting 2D.\\

Ada plot untuk menampilkan data secara 2D atau 3D. sehingga kamu dapat menampilkan data yang telah kamu olah sesuai kebutuhan. Matplotlib pun terintegrasi dengan ipython notebook atau jupyter dimana kamu dapat membuat sebuah buku interaktif yang dapat diberi penjelasan dan kode yang disisipkan begitupun hasil plottingnya.\\

\item 2. Jelaskan langkah-langkah membuat sumbu X dan Y di matplotlib.
A. Klik kanan scatter chart dan klik pilih data dalam menu konteks , lalu pilih "Select Data

B.Dalam kotak dialog sumber data dan sumber keluar , klik untuk menyoroti Y pada kolom. lalu klik tombol ubah pada tombol di legenda entrl.

C. Sekarang kotak dialog di edit pada series keluar , abis itu silahkan tukar Nilai seri X dan Nilai seri Y lalu klik tombol OK untuk menutup kedua kotak dialog.\\

\item 3. Jelaskan bagaimana perbedaan fungsi dan cara pakai untuk berbagai jenis(bar,histogram,scatter.dll) jenis plot di matloptlib
Untuk perbedaan fungsi plot yang digunakan adalah bentuk bentuk grafik yang akan di tampilkan sesuai dengan perintah yang digunakan pada pemogramannya.\\

line
    Perintah yang digunakan untuk membuat grafik line sebagai berikut.
x = [2,4,6,5,42,543,5,3,73,64,42,97,63,76,63,8,73,97,\\
23,45,56,89,45,3,23,2,5,78,23,56,67,78,8,3,78,34,67,\\
23,324,234,43,544,54,33,223,443,444,234,76,432,233,23,\\
232,243,222,221,254,222,276,300,353,354,387,364,309]\\



Bar itu di dalam Penggunaan plot bar koordinat x nya itu yang awal, dan untuk Y nya adalah yang kedua.\\

Histrogram itu di dalam penggunaan plot histogram titik x nya bisa tidak sama dengan titik Y. untuk penggunaannya bisa sebagai berikut.\\

scatter untuk penggunaa plot scatter atau bisa juga d bilang diagram titik.\\

Stack plot untuk penggunaan stack plot ini seperti diagram line, tapi ada fill colornya,jadi antar line itu bisa berdekatan.\\

\item 4. Jelaskan bagaimana cara menggunakan legend dan label serta kaitannya dengan fungsi tersebut.
Contoh source code lengkap disertai dengan link "editor" untuk mencoba (try it) dan melihat hasil (preview) kode.\\

Elemen yang akan ditambahkan ke legenda ditentukan secara otomatis, ketika Anda tidak memberikan argumen tambahan.\\

Garis-garis spesifik dapat dikecualikan dari pemilihan elemen legenda otomatis dengan mendefinisikan label dimulai dengan garis bawah.\\

\item 5. Jelaskan apa fungsi dari subplot di matplotlib dan fungsi dari subplot dari matplotlib untuk bisa membuat lebih dari 1 grafik dalam sebuah program.\\

Misalnya, kita dapat membuat sumbu inset di sudut kanan atas sumbu lain dengan mengatur posisi x dan y ke 0,65 yaitu, mulai dari 65 peren dari lebar dan 65 persen  dari ketinggian gambar dan x dan y meluas ke 0,2 yaitu, ukuran sumbu adalah 20 persen  dari lebar dan 20persen dari tinggi gambar.\\

Simple Grids of Subplots itu kebutuhan yang cukup umum sehingga Matplotlib memiliki beberapa rutinitas kenyamanan yang membuatnya mudah dibuat. Level terendah adalah plt.subplot (), yang membuat subplot tunggal di dalam kisi. Seperti yang Anda lihat, perintah ini membutuhkan tiga argumen bilangan bulat — jumlah baris, jumlah kolom, dan indeks plot yang akan dibuat dalam skema ini, yang berjalan dari kiri atas ke kanan bawah.\\
Contohnya:\\
for i in range(1, 7):
    plt.subplot(2, 3, i)
    plt.text(0.5, 0.5, str((2, 3, i)),
             fontsize=18, ha='center')
             
The Whole Grid in One Go itu  membuat grid besar subplot, terutama jika Anda ingin menyembunyikan label sumbu x dan y pada plot bagian dalam. Untuk tujuan ini, plt.subplots () adalah alat yang lebih mudah digunakan.\\
 
\item 6. Sebutkan semua parameter color yang bisa digunakan(contoh: m,c,r,k,...dkk)
Tipe Warna RGB
    Untuk keterangannya sebagai berikut
    R untuk warna Red atau Merah
    G untuk warna Green atau Hijau
    B untuk warna Blue atau Biru.\\
    
Tipe warna CMYK
    Untuk keterangannya sebagai berikut
    C untuk warna Cyan atau Biru Muda
    M untuk warna Mangenta atau Merah Tua
    Y untuk warna Yellow Atau Kuning
    K untuk warna blacK atau Hitam.\\

\item 7. Jelaskan bagaimana cara kerja dari fungsi hist , sertakan ilustrasi dan gambar sendiri.
Untuk fungsi histogram ini kedua titik koordinat boleh tidak sama. Misalnya x nya ada 10 nilai sedangkan Y nya ada 5 nilai, itu tidak akan jadi masalah karena diagram ini digunakan untuk mendata usia dari rentang tertentu atau kebutuhan lainnya.\\

Ini merupakan contoh dari penggunaan histogram.\\

\item 8. Jelaskan lebih dalam tentang parameter dari fungsi pie diantaranya labels , color , startangle , shadow , explode , autopct.
Jika jumlah x <1, maka nilai x memberikan area fraksional secara langsung dan array tidak akan dinormalisasi.\\

labels : Label digunakan untuk mempermudah pembaca dalam membaca diagram pie.\\

color : warna digunakan untuk membedakan antar data.\\

startangle : Digunakan untuk sudut yang digunakan untuk memulai diagram pie tersebut.\\

shadow :  bayangan digunakan untuk membuat bayangan dari setiap diagram pie yang menonjol.\\

explode : explode digunakan untuk mengeluarkan suatu data agar data tersebut terlihat menonjol.\\

autopct : Digunakan sesuai dengan berapa angka dibelakang koma yang kita inginkan.\\

\end{enumerate}
%%%%%%%%%%%%%%%%%%%%%%%%%%%%%%%%%%%%%%%%%%%%%%%%%%%%%%%%%%%%%%%%%%%%%%
\section{Advent Nopele Olasni Damiahan Sihite / 1174089}
\subsection{Teori}
\subsubsection{Soal No. 1}
\hfill \break
Apa itu fungsi library matplotlib?

\hfill \break
Matplotlib merupakan salah satu library Python 2D yang dapat menghasilkan plot dengan kualitas yang tinggi dalam berbagai format dan dapat digunakan di berbagai platform. Matplotlib berfungsi sebagai pembuat grafik di berbagai platform, seperti Python dan Jupyter. Grafik yang dibuat menggunakan Matplotlib bisa dibuat dalam berbagai bentuk, seperti grafik garis, batang, lingkaran, histogram, dan sebagainya.

\subsubsection{Soal No. 2}
\hfill \break
Jelaskan langkah-langkah membuat sumbu X dan Y di matplotlib!

\begin{enumerate}
	\item Pertama import library Matplotlib.	
	\lstinputlisting[firstline=2, lastline=2]{src/6/1174089/Teori/1174089.py}
	
	\item Buat variabel x yang menampung list untuk sumbu x dan variabel y yang menampung list untuk sumbu y.	
	\lstinputlisting[firstline=4, lastline=5]{src/6/1174089/Teori/1174089.py}
	
	\item Panggil fungsi plot dan isi parameter pertama dengan variabel x dan parameter kedua dengan variabel y.
	\lstinputlisting[firstline=7, lastline=7]{src/6/1174089/Teori/1174089.py}	

	\item Lalu panggil plot tadi dengan memanggil fungsi show.
	\lstinputlisting[firstline=9, lastline=9]{src/6/1174089/Teori/1174089.py}
	
\end{enumerate}
\hfill \break
\textbf{Kode Program}

\lstinputlisting[caption = Kode program membuat diagram menggunakan Matplotlib., firstline=2, lastline=9]{src/6/1174089/Teori/1174089.py}

\hfill \break
\textbf{Hasil Compile}

\begin{figure}[H]
	\includegraphics[width=12cm]{figures/6/1174089/Teori/2.png}
	\centering
	\caption{Hasil compile membuat diagram menggunakan Matplotlib.}
\end{figure}
 
\subsubsection{Soal No. 3}
\hfill \break
Jelaskan bagaimana perbedaan fungsi dan cara pakai untuk berbagai jenis(bar, histogram ,scatter ,line, dll) jenis plot di matplotlib!

\begin{enumerate}
	\item \textbf{Bar Graph}
	
	Perbedaan bar graph dengan jenis plot yang lain adalah bar graph menggunakan bar atau batang-batang untuk membandingkan data di antara berbagai kategori.
	
	\textbf{Kode Program}
	
	\lstinputlisting[caption = Kode program membuat bar graph menggunakan Matplotlib., firstline=2, lastline=9]{src/6/1174089/Teori/1174089.py}
	
	\textbf{Hasil Compile}
	
	\begin{figure}[H]
		\includegraphics[width=12cm]{figures/6/1174089/Teori/bar.png}
		\centering
		\caption{Hasil compile membuat bar graph menggunakan Matplotlib.}
	\end{figure}
	
	\item \textbf{Histogram}
	
	Perbedaan histogram dengan jenis plot yang lain adalah histogram akan membuat plot dimana plot yang dimunculkan merupakan gabungan dari beberapa data yang telah dikelompokkan.
	
	\textbf{Kode Program}
	
	\lstinputlisting[caption = Kode program membuat histogram menggunakan Matplotlib., firstline=29, lastline=36]{src/6/1174089/Teori/1174089.py}
	
	\textbf{Hasil Compile}
	
	\begin{figure}[H]
		\includegraphics[width=12cm]{figures/6/1174089/Teori/histogram.png}
		\centering
		\caption{Hasil compile membuat histogram menggunakan Matplotlib.}
	\end{figure}
	
	\item \textbf{Scatter Plot}
	
	Perbedaan scatter plot dengan jenis plot lain adalah scatter plot menampilkan data sebagai kumpulan titik, masing-masing memiliki nilai satu variabel yang menentukan posisi pada sumbu horizontal dan nilai variabel lain menentukan posisi pada sumbu vertikal.
	
	\textbf{Kode Program}
	
	\lstinputlisting[caption = Kode program membuat scatter plot menggunakan Matplotlib., firstline=40, lastline=53]{src/6/1174089/Teori/1174089.py}
	
	\textbf{Hasil Compile}
	
	\begin{figure}[H]
		\includegraphics[width=12cm]{figures/6/1174089/Teori/scatter.png}
		\centering
		\caption{Hasil compile membuat scatter plot menggunakan Matplotlib.}
	\end{figure}
	
	\item \textbf{Area Plot}
	
	Perbedaan area plot dengan jenis plot lain adalah area plot digunakan untuk melacak perubahan dari waktu ke waktu untuk dua atau lebih kelompok terkait yang membentuk satu kategori secara keseluruhan.
	
	\textbf{Kode Program}
	
	\lstinputlisting[caption = Kode program membuat diagram menggunakan Matplotlib., firstline=57, lastline=76]{src/6/1174089/Teori/1174089.py}
	
	\textbf{Hasil Compile}
	
	\begin{figure}[H]
		\includegraphics[width=12cm]{figures/6/1174089/Teori/area.png}
		\centering
		\caption{Hasil compile membuat diagram menggunakan Matplotlib.}
	\end{figure}
	
	\item \textbf{Pie Plot}
	
	Perbedaan pie plot dengan jenis plot lain adalah pie plot digunakan untuk menunjukkan persentase atau data proporsional di mana setiap potongan pie mewakili kategori.
	
	\textbf{Kode Program}
	
	\lstinputlisting[caption = Kode program membuat Pie Plot menggunakan Matplotlib., firstline=80, lastline=101]{src/6/1174089/Teori/1174089.py}
	
	\textbf{Hasil Compile}
	
	\begin{figure}[H]
		\includegraphics[width=9cm]{figures/6/1174089/Teori/pie.png}
		\centering
		\caption{Hasil compile membuat Pie Plot menggunakan Matplotlib.}
	\end{figure}
	
	\item \textbf{Line Graph}
	
	Perbedaan line graph dengan jenis plot lain adalah line graph menampilkan diagram dalam bentuk garis.
	
	\textbf{Kode Program}
	
	\lstinputlisting[caption = Kode program membuat diagram menggunakan Matplotlib., firstline=105, lastline=113]{src/6/1174089/Teori/1174089.py}
	
	\textbf{Hasil Compile}
	
	\begin{figure}[H]
		\includegraphics[width=12cm]{figures/6/1174089/Teori/line.png}
		\centering
		\caption{Hasil compile membuat diagram menggunakan Matplotlib.}
	\end{figure}
	
\end{enumerate}

\subsubsection{Soal No. 4}
\hfill \break
Jelaskan bagaimana cara menggunakan legend dan label serta kaitannya dengan fungsi tersebut!

\begin{enumerate}
	\item Untuk menggunakan legend definisikan parameter label di tiap fungsi plot. Parameter label digunakan untuk memberikan label pada line sebagai pembeda antar line.
	
	\lstinputlisting[caption = Kode program menggunakan parameter label dengan Matplotlib., firstline=123, lastline=124]{src/6/1174089/Teori/1174089.py}
	
	\item Kemudian panggil fungsi legend.
	
	\lstinputlisting[caption = Kode program memanggil fungsi legend dengan Matplotlib., firstline=128, lastline=128]{src/6/1174089/Teori/1174089.py}
\end{enumerate}

\hfill \break
\textbf{Kode Program}

\lstinputlisting[caption = Kode program membuat diagram menggunakan Matplotlib., firstline=117, lastline=130]{src/6/1174089/Teori/1174089.py}

\hfill \break
\textbf{Hasil Compile}

\begin{figure}[H]
	\includegraphics[width=12cm]{figures/6/1174089/Teori/4.png}
	\centering
	\caption{Hasil compile membuat diagram menggunakan Matplotlib.}
\end{figure}

\subsubsection{Soal No. 5}
\hfill \break
Jelaskan apa fungsi dari subplot di matplotlib, dan bagaimana cara kerja dari fungsi subplot, sertakan ilustrasi dan gambar sendiri dan apa parameternya jika ingin menggambar plot dengan 9 subplot di dalamnya!

\hfill \break
Fungsi subplot adalah untuk membuat beberapa plot di dalam satu gambar.
\hfill \break
Cara kerja subplot, yaitu fungsi subplot memiliki parameter pertama adalah jumlah kolom, parameter kedua adalah jumlah baris, dan parameter ketiga adalah index plot keberapanya.

\hfill \break
\textbf{Kode Program}

\lstinputlisting[caption = Kode program membuat subplot menggunakan Matplotlib., firstline=134, lastline=146]{src/6/1174089/Teori/1174089.py}

\hfill \break
\textbf{Hasil Compile}

\begin{figure}[H]
	\includegraphics[width=12cm]{figures/6/1174089/Teori/subplot.png}
	\centering
	\caption{Hasil compile membuat subplot menggunakan Matplotlib.}
\end{figure}

\subsubsection{Soal No. 6}
\hfill \break
Sebutkan semua parameter color yang bisa digunakan (contoh:  m,c,r,k,...  dkk)!

\begin{itemize}
	\item 'b' (blue)
	\item 'g' (green)
	\item 'r' (red)
	\item 'c' (cyan)
	\item 'm' (magenta)
	\item 'y' (yellow)
	\item 'k' (black)
	\item 'w' (white)
\end{itemize}

\subsubsection{Soal No. 7}
\hfill \break
Jelaskan bagaimana cara kerja dari fungsi hist, sertakan ilustrasi dan gambar sendiri!

\hfill \break
Cara kerja dari fungsi hist yaitu fungsi hist akan menerima parameter yang diberikan, kemudian fungsi hist akan dieksekusi sesuai dengan parameter yang diberikan.

\hfill \break
\textbf{Kode Program}

\lstinputlisting[caption = Kode program membuat diagram menggunakan Matplotlib., firstline=150, lastline=157]{src/6/1174089/Teori/1174089.py}

\hfill \break
\textbf{Hasil Compile}

\begin{figure}[H]
	\includegraphics[width=12cm]{figures/6/1174089/Teori/histogram.png}
	\centering
	\caption{Hasil compile membuat diagram menggunakan Matplotlib.}
\end{figure}

\subsubsection{Soal No. 8}
\hfill \break
 Jelaskan lebih mendalam tentang parameter dari fungsi pie diantaranya labels, colors, startangle, shadow, explode, autopct!
 
 \begin{itemize}
 	\item labels : untuk memberikan label di tiap persentase.
 	\item colors : untuk memberikan warna di tiap persentase.
 	\item startangle : untuk memutar plot sesuai dengan derajat yang ditentukan.
 	\item shadow : untuk memberikan bayangan pada plot.
 	\item explode : untuk memisahkan antar tiap potongan pie pada plot.
 	\item autopct : untuk menentukan jumlah angka dibelakang koma.
 \end{itemize}

\subsection{Praktek}
\subsubsection{Soal No. 1}
\hfill \break
Buatlah librari fungsi (file terpisah/library dengan nama NPMbar.py) untuk plot dengan jumlah subplot adalah NPM mod 3 + 2!

\subsubsection{Soal No. 2}
\hfill \break
Buatlah librari fungsi (file terpisah/library dengan nama NPMscatter.py) untuk plot dengan jumlah subplot NPM mod 3 + 2!

\subsubsection{Soal No. 3}
\hfill \break
Buatlah librari fungsi (file terpisah/library dengan nama NPMpie.py) untuk plot dengan jumlah subplot NPM mod 3 + 2!

\subsubsection{Soal No. 4}
\hfill \break
Buatlah librari fungsi (file terpisah/library dengan nama NPMplot.py) untuk plot dengan jumlah subplot NPM mod 3 + 2


\subsection{Penanganan Error}
Tuliskan  peringatan  error  yang  didapat  dari  mengerjakan  praktek  keenam  ini, dan  jelaskan  cara  penanganan  error  tersebut. dan  Buatlah  satu  fungsi  yang menggunakan try except untuk menanggulangi error tersebut.
%%%%%%%%%%%%%%%%%%%%%%%%%%%%%%%%%%%%%%%%%%%%%%%%%%%%%%%%%%%%%